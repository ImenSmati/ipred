\documentclass[11pt]{article}
\usepackage[round]{natbib}
\usepackage{bibentry}
\usepackage{amsfonts}
\usepackage{hyperref}
\renewcommand{\baselinestretch}{1.3}
\newcommand{\ipred}{\texttt{ipred }}

%%\VignetteIndexEntry{Some more or less useful examples for illustration.}
%%\VignetteDepends{ipred}


\usepackage{/usr/local/lib/R/share/texmf/Sweave}
\begin{document}
\title{\ipred: Improved Predictors}
\date{}



\maketitle

This short manual is heavily based on
\cite{Rnews:Peters+Hothorn+Lausen:2002} and needs some improvements.

\section{Introduction}
In classification problems, there are several attempts to create rules which assign future observations to 
certain classes. Common methods are for 
example linear discriminant analysis or 
classification trees. Recent developments lead to substantial reduction of misclassification error 
in many applications. 
Bootstrap aggregation \citep[``bagging'',][]{breiman:1996} combines 
classifiers trained on bootstrap samples of the original data. Another
approach is indirect classification, which 
incorporates a priori knowledge 
into a classification rule \citep{hand:2001}.
Since the misclassification error is a criterion to assess the 
classification techniques, its estimation is of main importance. 
A nearly unbiased but highly variable estimator can be calculated by cross validation. \cite{efron:1997} discuss bootstrap 
estimates of misclassification error.
As a by-product of bagging, \cite{out-of-bag:1996} proposes the out-of-bag 
estimator. \\ 
However, the calculation of the desired classification models and 
their misclassification errors is often aggravated by different and
specialized interfaces of the various procedures. We propose the \ipred
package as a first attempt to create a unified interface for improved predictors and various error rate estimators. 
In the following we demonstrate the functionality of the package 
in the example of glaucoma classification. We start with an overview 
about the disease and data and review the implemented 
classification and estimation methods in context with their
application to glaucoma diagnosis.
  

\section{Glaucoma}
Glaucoma is a slowly processing and irreversible disease that affects 
the optic nerve head. It is the second most reason for blindness worldwide. 
Glaucoma is usually diagnosed based on a reduced visual field, 
assessed by a medical examination of perimetry and a smaller number of 
intact nerve fibers at the optic nerve head. One opportunity to examine 
the amount of intact nerve fibers is using the Heidelberg Retina 
Tomograph (HRT), a confocal laser scanning tomograph, which does a 
three dimensional topographical analysis of the optic nerve head morphology. 

It produces a series of $32$ images, each of $256 \times 256$ pixels, 
which are converted to a single topographic image. A less complex, 
but although a less informative examination tool is the $2$-dimensional 
fundus photography. However, in cooperation with clinicians and a 
priori analysis we derived a diagnosis of glaucoma based on three variables 
only: $w_{lora}$ represents the loss of nerve fibers and is obtained by a
$2$-dimensional fundus photography, $w_{cs}$ and $w_{clv}$ describe the 
visual field defect \citep{ifcs:2001}. 

\begin{center}
\begin{figure}[h]
\begin{center}
{\small
\setlength{\unitlength}{0.6cm}
\begin{picture}(14.5,5)
    \put(5, 4.5){\makebox(2, 0.5){$w_{clv}\geq 5.1$}}
        \put(2.5, 3){\makebox(2, 0.5){$w_{lora}\geq 49.23$}}
          \put(7.5, 3){\makebox(2, 0.5){$w_{lora} \geq 58.55$}}
\put(0, 1.5){\makebox(2, 0.5){$glaucoma$}}
    \put(3.5, 1.5){\makebox(2, 0.5){$normal$}}
       \put(6.5, 1.5){\makebox(2, 0.5){$w_{cs} < 1.405$}}
          \put(10, 1.5){\makebox(2, 0.5){$normal$}}

      \put(3.5, 0){\makebox(2, 0.5){$glaucoma$}}
       \put(6.5, 0){\makebox(2, 0.5){$normal$}}

    \put(6, 4.5){\vector(-3, -2){1.5}}
    \put(6, 4.5){\vector(3, -2){1.5}}

  \put(3.5, 3){\vector(3, -2){1.5}}
  \put(3.5, 3){\vector(-3, -2){1.5}}
        \put(8.5, 3){\vector(3, -2){1.5}}
        \put(8.5, 3){\vector(-3, -2){1.5}}

      \put(6.5, 1.5){\vector(3, -2){1.5}}
      \put(6.5, 1.5){\vector(-3, -2){1.5}}
\end{picture}
}
\end{center}
\caption{Glaucoma diagnosis. \label{diag}}
\end{figure}
\end{center}

Figure \ref{diag} represents the diagnosis of glaucoma in terms of a medical 
decision tree. A complication of the disease is that a damage in the 
optic nerve head morphology precedes a measurable 
visual field defect. Furthermore, an early detection 
is of main importance, since an adequate therapy can only slow down the 
progression of the disease. Hence, a classification rule for detecting 
early damages should include morphological informations, rather than 
visual field data only. 

Two example datasets are included in the package. The first one contains
measurements of the eye morphology only (\texttt{GlaucomaM}), including $62$
variables for $196$ observations. The second dataset (\texttt{GlaucomaMVF})
contains additional visual field measurements for a different set of
patients. In both example datasets, the observations in the two groups are
matched by age and sex to prevent any bias.

\section{Bagging}
Referring to the example of glaucoma diagnosis we first 
demonstrate the functionality of the \texttt{bagging} function. 
We fit \texttt{nbagg = 25} (default) classification trees for bagging by 
\begin{Sinput}
>library(ipred)