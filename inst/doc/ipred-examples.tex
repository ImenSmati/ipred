\documentclass[11pt]{article}
\usepackage[round]{natbib}
\usepackage{bibentry}
\usepackage{amsfonts}
\usepackage{hyperref}
\renewcommand{\baselinestretch}{1.3}
\newcommand{\ipred}{\texttt{ipred }}

%\VignetteIndexEntry{Some more or less useful examples for illustration.}
%\VignetteDepends{ipred}
%\textwidth=6.2in 
%\VignetteDepends{mvtnorm}

\usepackage{/usr/local/lib/R/share/texmf/Sweave}
\begin{document}
\title{\ipred: Improved Predictors}
\date{}



\maketitle

This short manual is heavily based on
\cite{Rnews:Peters+Hothorn+Lausen:2002} and needs some improvements.

\section{Introduction}
In classification problems, there are several attempts to create rules which assign future observations to 
certain classes. Common methods are for 
example linear discriminant analysis or 
classification trees. Recent developments lead to substantial reduction of misclassification error 
in many applications. 
Bootstrap aggregation \citep[``bagging'',][]{breiman:1996} combines 
classifiers trained on bootstrap samples of the original data. Another
approach is indirect classification, which 
incorporates a priori knowledge 
into a classification rule \citep{hand:2001}.
Since the misclassification error is a criterion to assess the 
classification techniques, its estimation is of main importance. 
A nearly unbiased but highly variable estimator can be calculated by cross validation. \cite{efron:1997} discuss bootstrap 
estimates of misclassification error.
As a by-product of bagging, \cite{out-of-bag:1996} proposes the out-of-bag 
estimator. \\ 
However, the calculation of the desired classification models and 
their misclassification errors is often aggravated by different and
specialized interfaces of the various procedures. We propose the \ipred
package as a first attempt to create a unified interface for improved predictors and various error rate estimators. 
In the following we demonstrate the functionality of the package 
in the example of glaucoma classification. We start with an overview 
about the disease and data and review the implemented 
classification and estimation methods in context with their
application to glaucoma diagnosis.
  

\section{Glaucoma}
Glaucoma is a slowly processing and irreversible disease that affects 
the optic nerve head. It is the second most reason for blindness worldwide. 
Glaucoma is usually diagnosed based on a reduced visual field, 
assessed by a medical examination of perimetry and a smaller number of 
intact nerve fibers at the optic nerve head. One opportunity to examine 
the amount of intact nerve fibers is using the Heidelberg Retina 
Tomograph (HRT), a confocal laser scanning tomograph, which does a 
three dimensional topographical analysis of the optic nerve head morphology. 

It produces a series of $32$ images, each of $256 \times 256$ pixels, 
which are converted to a single topographic image. A less complex, 
but although a less informative examination tool is the $2$-dimensional 
fundus photography. However, in cooperation with clinicians and a 
priori analysis we derived a diagnosis of glaucoma based on three variables 
only: $w_{lora}$ represents the loss of nerve fibers and is obtained by a
$2$-dimensional fundus photography, $w_{cs}$ and $w_{clv}$ describe the 
visual field defect \citep{ifcs:2001}. 

\begin{center}
\begin{figure}[h]
\begin{center}
{\small
\setlength{\unitlength}{0.6cm}
\begin{picture}(14.5,5)
    \put(5, 4.5){\makebox(2, 0.5){$w_{clv}\geq 5.1$}}
        \put(2.5, 3){\makebox(2, 0.5){$w_{lora}\geq 49.23$}}
          \put(7.5, 3){\makebox(2, 0.5){$w_{lora} \geq 58.55$}}
\put(0, 1.5){\makebox(2, 0.5){$glaucoma$}}
    \put(3.5, 1.5){\makebox(2, 0.5){$normal$}}
       \put(6.5, 1.5){\makebox(2, 0.5){$w_{cs} < 1.405$}}
          \put(10, 1.5){\makebox(2, 0.5){$normal$}}

      \put(3.5, 0){\makebox(2, 0.5){$glaucoma$}}
       \put(6.5, 0){\makebox(2, 0.5){$normal$}}

    \put(6, 4.5){\vector(-3, -2){1.5}}
    \put(6, 4.5){\vector(3, -2){1.5}}

  \put(3.5, 3){\vector(3, -2){1.5}}
  \put(3.5, 3){\vector(-3, -2){1.5}}
        \put(8.5, 3){\vector(3, -2){1.5}}
        \put(8.5, 3){\vector(-3, -2){1.5}}

      \put(6.5, 1.5){\vector(3, -2){1.5}}
      \put(6.5, 1.5){\vector(-3, -2){1.5}}
\end{picture}
}
\end{center}
\caption{Glaucoma diagnosis. \label{diag}}
\end{figure}
\end{center}

Figure \ref{diag} represents the diagnosis of glaucoma in terms of a medical 
decision tree. A complication of the disease is that a damage in the 
optic nerve head morphology precedes a measurable 
visual field defect. Furthermore, an early detection 
is of main importance, since an adequate therapy can only slow down the 
progression of the disease. Hence, a classification rule for detecting 
early damages should include morphological informations, rather than 
visual field data only. 

Two example datasets are included in the package. The first one contains
measurements of the eye morphology only (\texttt{GlaucomaM}), including $62$
variables for $196$ observations. The second dataset (\texttt{GlaucomaMVF})
contains additional visual field measurements for a different set of
patients. In both example datasets, the observations in the two groups are
matched by age and sex to prevent any bias.

\section{Bagging}
Referring to the example of glaucoma diagnosis we first 
demonstrate the functionality of the \texttt{bagging} function. 
We fit \texttt{nbagg = 25} (default) classification trees for bagging by 
\begin{Schunk}
\begin{Sinput}
>library(ipred)
\end{Sinput}
\begin{Soutput}
Loading required package: rpart 
Loading required package: MASS 
Loading required package: mlbench 
Loading required package: survival 
Loading required package: splines 
Loading required package: nnet 
Loading required package: class 
\end{Soutput}
\begin{Sinput}
>data(GlaucomaM)
>gbag <- bagging(Class ~ ., data = GlaucomaM, 
+     coob = TRUE)
\end{Sinput}
\end{Schunk}
where \texttt{GlaucomaM} contains explanatory HRT variables 
and the response of glaucoma diagnosis (\texttt{Class}), 
a factor at two levels \texttt{normal} and \texttt{glaucoma}.
\texttt{print} returns informations about the returned object,
i.e. the number of bootstrap replications used and, as requested by
\texttt{coob=TRUE}, the out-of-bag estimate of misclassification error 
\citep{out-of-bag:1996}. 
\begin{Schunk}
\begin{Sinput}
>print(gbag)
\end{Sinput}
\begin{Soutput}
Bagging classification trees with 25 bootstrap replications 

Call: bagging.data.frame(formula = Class ~ ., data = GlaucomaM, coob = TRUE)

Out-of-bag estimate of misclassification error:  0.1684 
\end{Soutput}
\end{Schunk}
The out-of-bag estimate uses the observations which are left out in a
bootstrap sample to estimate the misclassification error at almost no
additional computational costs. 
\cite{double-bag:2002} propose to use the
out-of-bag samples for a combination of linear discriminant analysis and
classification trees, called ``Double-Bagging''. For example, a combination
of a stabilised linear disciminant analysis with classification trees can be
computed along the following lines
\begin{Schunk}
\begin{Sinput}
>scomb <- list(list(model = slda, predict = function(object, 
+     newdata) predict(object, newdata)$x))
>gbagc <- bagging(Class ~ ., data = GlaucomaM, 
+     comb = scomb)
\end{Sinput}
\end{Schunk}
\texttt{predict} predicts future observations according to the 
fitted model.
\begin{Schunk}
\begin{Sinput}
>predict(gbagc, newdata = GlaucomaM[c(1:3, 
+     99:102), ])
\end{Sinput}
\begin{Soutput}
[1] normal   normal   normal   glaucoma glaucoma
[6] glaucoma glaucoma
Levels: glaucoma normal
\end{Soutput}
\end{Schunk}
Both \texttt{bagging} and \texttt{predict} rely on the \texttt{rpart}
routines.  The \texttt{rpart} routine for each bootstrap sample
can be controlled in the usual way. By default \texttt{rpart.control} is used
with \texttt{minsize=2} and \texttt{cp=0} and it is wise to turn
cross-validation off (\texttt{xval=0}). The function \texttt{prune} can
be used to prune each of the trees to an
appropriate size.

\section{Indirect Classification}
Especially in a medical context it often occurs that a priori 
knowledge about a classifying structure is given. For example 
it might be known that a disease is assessed on a subgroup of 
the given variables or, moreover, that class memberships are 
assigned by a deterministically known classifying function. 
\cite{hand:2001} proposes the framework of indirect classification 
which incorporates this a priori knowledge into a classification rule. 
In this framework we subdivide a given data set into three groups of 
variables: those to be used  predicting the class membership 
(explanatory), those to be used defining the class membership 
(intermediate) and the class membership variable itself (response). 
For future observations, an indirect classifier predicts values 
for the appointed intermediate variables based
on explanatory variables only. The observation is classified 
based on their predicted intermediate variables and a fixed 
classifying function. This indirect way of classification using 
the predicted intermediate variables offers possibilities to 
incorporate a priori knowledge by the subdivision of variables and 
by the construction of a fixed classifying function.

We apply indirect classification by using the function \texttt{inclass}.
Referring to the glaucoma example, explanatory variables are HRT 
and anamnestic variables only, intermediate variables 
are $w_{lora}, \, w_{cs}$ and $w_{clv}$. The response is the 
diagnosis of glaucoma which is determined by a fixed classifying 
function and therefore not included in the learning 
sample \texttt{GlaucomaMVF}. We assign the given variables to explanatory 
and intermediate by specifying the input formula.
\begin{Schunk}
\begin{Sinput}
>data(GlaucomaMVF)
>GlaucomaMVF <- GlaucomaMVF[, -63]
>formula.indirect <- Class ~ clv + lora + 
+     cs ~ .
\end{Sinput}
\end{Schunk}
The variables on the left-hand side represent the intermediate variables, 
modeled by the explanatory variables on the right-hand side. Almost each 
modeling technique can be used to predict the intermediate variables. We 
chose a linear model by \texttt{pFUN = list(list(model = lm))}.
\begin{Schunk}
\begin{Sinput}
>classify <- function(data) {
+     attach(data)
+     res <- ifelse((!is.na(clv) & !is.na(lora) & 
+         clv >= 5.1 & lora >= 49.23372) | 
+         (!is.na(clv) & !is.na(lora) & 
+             !is.na(cs) & clv < 5.1 & lora >= 
+             58.55409 & cs < 1.405) | (is.na(clv) & 
+         !is.na(lora) & !is.na(cs) & lora >= 
+         58.55409 & cs < 1.405) | (!is.na(clv) & 
+         is.na(lora) & cs < 1.405), 0, 
+         1)
+     detach(data)
+     factor(res, labels = c("glaucoma", 
+         "normal"))
+ }
>fit <- inclass(formula.indirect, pFUN = list(list(model = lm)), 
+     cFUN = classify, data = GlaucomaMVF)
\end{Sinput}
\end{Schunk}
\texttt{print} displays the subdivision of variables and the chosen 
modeling technique
\begin{Schunk}
\begin{Sinput}
>print(fit)
\end{Sinput}
\begin{Soutput}
 Indirect classification, with 3 intermediate variables: 
 clv lora cs 
 
 Predictive model per intermediate is lm 
\end{Soutput}
\end{Schunk}
Furthermore, indirect classification predicts the intermediate 
variables based on the explanatory variables and classifies them 
according to a fixed classifying function in a second step, that means 
a deterministically known function for the class membership has to be 
specified. In our example this function is given in 
Figure \ref{diag} and implemented in the function \texttt{classify}.\\
Prediction of future observations is now performed by
\begin{Schunk}
\begin{Sinput}
>predict(object = fit, newdata = GlaucomaMVF[c(1:3, 
+     86:88), ])
\end{Sinput}
\begin{Soutput}
[1] normal   normal   normal   glaucoma glaucoma
[6] glaucoma
Levels: glaucoma normal
\end{Soutput}
\end{Schunk}
We perform a bootstrap aggregated indirect classification approach by 
choosing \texttt{pFUN = bagging} and specifying the number of 
bootstrap samples \citep{ifcs:2001}. Regression or classification 
trees are fitted for each bootstrap sample, with respect to the 
measurement scale of the specified intermediate variables 
\begin{Schunk}
\begin{Sinput}
>mypredict.rpart <- function(object, newdata) {
+     RES <- predict(object, newdata)
+     RET <- rep(NA, nrow(newdata))
+     NAMES <- rownames(newdata)
+     RET[NAMES %in% names(RES)] <- RES[NAMES[NAMES %in% 
+         names(RES)]]
+     RET
+ }
>fit <- inbagg(formula.indirect, pFUN = list(list(model = rpart, 
+     predict = mypredict.rpart)), cFUN = classify, 
+     nbagg = 25, data = GlaucomaMVF)
\end{Sinput}
\end{Schunk}
The call for the prediction of values remains unchanged.


\section{Error Rate Estimation}
Classification rules are usually assessed by their misclassification rate. 
Hence, error rate estimation is of main importance. 
The function \texttt{errorest} implements a unified interface to several
resampling based estimators. Referring to the example, we apply a linear
discriminant analysis and specify the error rate estimator 
by \texttt{estimator = "cv", "boot"} or \texttt{"632plus"}, 
respectively. A 10-fold cross validation is performed by 
choosing \texttt{estimator = "cv"} and 
\texttt{est.para = control.errorest(k = 10)}. The options \texttt{estimator = "boot"} or 
\texttt{estimator = "632plus"} deliver a bootstrap estimator 
and its  bias corrected version {\sl .632+} \citep[see][]{efron:1997}, 
we specify the number of bootstrap samples to be drawn by 
\texttt{est.para = control.errorest(nboot = 50)}. 
Further arguments are required to particularize the 
classification technique. The argument \texttt{predict} represents 
the chosen predictive function. For a unified interface 
\texttt{predict} has to be based on the arguments \texttt{object} 
and \texttt{newdata} only, therefore a wrapper function \texttt{mypredict} is necessary for classifiers 
which require more than those arguments or do not return the predicted
classes by default. For a linear discriminant analysis with \texttt{lda}, we
need to specify  
\begin{Schunk}
\begin{Sinput}
>mypredict.lda <- function(object, newdata) {
+     predict(object, newdata = newdata)$class
+ }
\end{Sinput}
\end{Schunk}
and calculate a 10-fold-cross-validated error rate estimator 
for a linear discriminant analysis by calling 
\begin{Schunk}
\begin{Sinput}
>errorest(Class ~ ., data = GlaucomaM, 
+     model = lda, estimator = "cv", predict = mypredict.lda)
\end{Sinput}
\begin{Soutput}
Call:
errorest.data.frame(formula = Class ~ ., data = GlaucomaM, model = lda, 
    predict = mypredict.lda, estimator = "cv")

	 10-fold cross-validation estimator of misclassification error 

Misclassification error:  0.2041 
\end{Soutput}
\end{Schunk}
For the indirect approach the specification of the call becomes 
slightly more complicated. 
%Again for a unified interface a wrapper 
%function has to be used, which incorporates the fixed classification rule
The bias corrected estimator {\sl .632+} is computed by 
\begin{Schunk}
\begin{Sinput}
>errorest(formula.indirect, data = GlaucomaMVF, 
+     model = inclass, estimator = "632plus", 
+     pFUN = list(list(model = lm)), cFUN = classify)
\end{Sinput}
\begin{Soutput}
Call:
errorest.data.frame(formula = formula.indirect, data = GlaucomaMVF, 
    model = inclass, estimator = "632plus", pFUN = list(list(model = lm)), 
    cFUN = classify)

	 .632+ Bootstrap estimator of misclassification error 
	 with 25 bootstrap replications

Misclassification error:  0.2417 
\end{Soutput}
\end{Schunk}
Because of the subdivision of variables and a formula describing the 
modeling between explanatory and intermediate variables only, 
we must call the class membership variable. Hence, in contrast to the 
function \texttt{inclass} the data set \texttt{GlaucomaMVF} used in 
\texttt{errorest} must contain explanatory, intermediate and response 
variables. 

Sometimes it may be necessary to reduce the number of predictors before
training a classifier. Estimating the error rate after the variable
selection leads to biased estimates of the misclassfication error and
therefore one should estimate the error rate of the whole procedure. Within
the \texttt{errorest} framework, this can be done as follows. First, we define
a function which does both variable selection and training of the
classifier. For illustration proposes, we select the predictors by comparing
their univariate $P$-values of a two-sample $t$-test with a prespecified
level and train a LDA using the selected variables only.

\begin{Schunk}
\begin{Sinput}
>mymod <- function(formula, data, level = 0.05) {
+     sel <- which(lapply(data, function(x) {
+         if (!is.numeric(x)) 
+             return(1)
+         else return(t.test(x ~ data$Class)$p.value)
+     }) < level)
+     sel <- c(which(colnames(data) %in% 
+         "Class"), sel)
+     mod <- lda(formula, data = data[, 
+         sel])
+     function(newdata) {
+         predict(mod, newdata = newdata[, 
+             sel])$class
+     }
+ }
\end{Sinput}
\end{Schunk}

Note that \texttt{mymod} does not return 
an object of class \texttt{lda} but a function
with argument \texttt{newdata} only. Thanks to lexical scoping, this 
function is used for computing
predicted classes instead of a function \texttt{predict} passed to
\texttt{errorest} as argument. Computing a $5$-fold cross-validated error rate
estimator now is approximately a one-liner.

\begin{Schunk}
\begin{Sinput}
>errorest(Class ~ ., data = GlaucomaM, 
+     model = mymod, estimator = "cv", est.para = control.errorest(k = 5))
\end{Sinput}
\begin{Soutput}
Call:
errorest.data.frame(formula = Class ~ ., data = GlaucomaM, model = mymod, 
    estimator = "cv", est.para = control.errorest(k = 5))

	 5-fold cross-validation estimator of misclassification error 

Misclassification error:  0.2602 
\end{Soutput}
\end{Schunk}


%%To summarize the performance of the different classification techniques in the considered example of glaucoma diagnosis, the 10-fold 
%%cross-validated error estimator delivers the 
%%results given in Table \ref{tenf}.
%%\begin{figure}
%%\begin{center}
%%\begin{tabular}{  rrr  }
%%\hline
%%dataset & method & error estimate \\
%%\hline
%%\texttt{GlaucomaM} & {\sl slda} & 0.168 \\
%%\texttt{GlaucomaM} & {\sl bagging} & 0.158 \\
%%\texttt{GlaucomaM} & {\sl double-bagging} & 0.153 \\
%%\texttt{GlaucomaMVF} & {\sl inclass-bagging} & 0.206 \\
%%\tetxtt{GlaucomaMVF} & {\sl inclass-lm} & 0.229 \\
%%\hline
%%\end{tabular}
%%\caption{10-fold cross-validated error estimation of 
%%the misclassification error for several classification 
%%methods: {\sl slda} - stabilised linear discriminant analysis, 
%%{\sl bagging} - bagging with 50 bootstrap samples, 
%%{\sl double-bagging} - bagging with 50 bootstrap samples, 
%%combined with sLDA, {\sl inclass-bagging} - 
%%indirect classification using bagging, 
%%{\sl inclass-lm} indirect classification using 
%%linear modeling. \label{tenf}}
%%\end{center}
%%\end{figure}
%%Note that an estimator of the variance is available for the ordinary 
%%bootstrap estimator (\texttt{estimator="boot"}) only, see \cite{efron:1997}.


\section{Summary}
\ipred tries to implement a unified interface to some recent developments
in classification and error rate estimation. It is by no means finished
nor perfect and we very much appreciate comments, suggestions and criticism.
Currently, the major drawback is speed. Calling \texttt{rpart} $50$ 
times for each bootstrap sample is relatively inefficient 
but the design of interfaces was our main focus instead of optimization.
Beside the examples shown, \texttt{bagging} can be used to compute bagging
for regression trees and \texttt{errorest} computes estimators of the 
mean squared error for regression models.

\bibliographystyle{plainnat}
\bibliography{ipred}


\end{document}
